\documentclass[a4paper,10pt]{article}
\usepackage{amsmath,amsfonts}
\usepackage[left=1cm,right=1cm,top=1cm,bottom=1cm]{geometry}
\usepackage{longtable}
\usepackage{booktabs}
\begin{document}
\centering
\begin{longtable}{cccccccccccccc}
\toprule
 & RA & PA & GAMQ & TAU & NU & PSIP & PSIY & RHOR & RHOG & RHOZ & SIGR & SIGG & SIGZ \\
\midrule
300/100 & 1.564 & 1.087 & 1.264 & 2.004 & 2.636 & 1.393 & 1.178 & 2.319 & 7.485 & 1.639 & 2.856 & 3.011 & 1.788 \\
900/300 & 1.725 & 2.462 & 1.950 & 1.277 & 1.727 & 3.865 & 6.341 & 3.448 & 3.025 & 3.268 & 3.694 & 3.471 & 2.784 \\
2700/900 & 2.766 & 2.894 & 2.560 & 2.784 & 3.051 & 2.745 & 2.922 & 2.977 & 2.647 & 3.157 & 3.108 & 3.101 & 3.100 \\
8100/2700 & 3.015 & 2.883 & 3.111 & 3.054 & 3.420 & 3.447 & 3.609 & 2.975 & 2.763 & 2.816 & 2.834 & 2.926 & 2.976 \\
\bottomrule
\caption{Bayesian Weak Identification An Schorfheide Convergence Ratiosmcmc method}
\label{table:tbl:WeakAnSchoConvergenceRatios_mcmc}
\end{longtable}
\end{document}
