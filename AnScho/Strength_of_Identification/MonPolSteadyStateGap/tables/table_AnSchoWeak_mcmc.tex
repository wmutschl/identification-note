\documentclass[a4paper,10pt]{article}
\usepackage{amsmath,amsfonts}
\usepackage[left=1cm,right=1cm,top=1cm,bottom=1cm]{geometry}
\usepackage{longtable}
\usepackage{booktabs}
\begin{document}
\centering
\begin{longtable}{cccccccccccccc}
\toprule
 & RA & PA & GAMQ & TAU & NU & PSIP & PSIY & RHOR & RHOG & RHOZ & SIGR & SIGG & SIGZ \\
\midrule
100 & 0.075 & 0.025 & 0.523 & 0.070 & 9.020 & 0.693 & 1.831 & 11.527 & 4.703 & 39.077 & 36.586 & 2.246 & 9.015 \\
300 & 0.039 & 0.009 & 0.221 & 0.047 & 7.927 & 0.322 & 0.719 & 8.912 & 11.735 & 21.345 & 34.834 & 2.254 & 5.373 \\
900 & 0.022 & 0.007 & 0.143 & 0.020 & 4.564 & 0.415 & 1.519 & 10.241 & 11.832 & 23.252 & 42.893 & 2.608 & 4.985 \\
2700 & 0.021 & 0.007 & 0.122 & 0.019 & 4.641 & 0.379 & 1.480 & 10.163 & 10.438 & 24.466 & 44.439 & 2.696 & 5.151 \\
8100 & 0.021 & 0.007 & 0.127 & 0.019 & 5.292 & 0.436 & 1.780 & 10.079 & 9.613 & 22.966 & 41.976 & 2.630 & 5.109 \\
\bottomrule
\caption{Bayesian Weak Identification An Schorfheide mcmc method}
\label{table:tbl:WeakAnScho_mcmc}
\end{longtable}
\end{document}
