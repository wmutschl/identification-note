\documentclass[a4paper,10pt]{article}
\usepackage{amsmath,amsfonts}
\usepackage[left=1cm,right=1cm,top=1cm,bottom=1cm]{geometry}
\usepackage{longtable}
\usepackage{booktabs}
\begin{document}
\centering
\begin{longtable}{ccccccccccccccc}
\toprule
 & RA & PA & GAMQ & TAU & NU & PSIP & PSIY & RHOR & RHOG & RHOZ & SIGR & SIGG & SIGZ & IOTAP \\
\midrule
100 & 0.074 & 0.041 & 0.494 & 0.077 & 12.615 & 0.539 & 0.325 & 12.077 & 3.268 & 32.418 & 35.518 & 4.083 & 7.383 & 1.083 \\
300 & 0.042 & 0.022 & 0.234 & 0.051 & 11.921 & 0.314 & 0.125 & 8.045 & 8.271 & 23.608 & 28.648 & 5.308 & 6.991 & 0.632 \\
900 & 0.025 & 0.012 & 0.153 & 0.027 & 7.086 & 0.311 & 0.084 & 6.286 & 9.069 & 23.826 & 32.700 & 5.472 & 5.346 & 0.580 \\
2700 & 0.024 & 0.011 & 0.131 & 0.030 & 7.039 & 0.225 & 0.051 & 4.597 & 8.245 & 27.163 & 28.740 & 5.616 & 6.179 & 0.589 \\
8100 & 0.021 & 0.010 & 0.119 & 0.030 & 7.507 & 0.203 & 0.042 & 3.772 & 8.074 & 26.849 & 25.771 & 5.590 & 6.179 & 0.584 \\
\bottomrule
\caption{Bayesian Weak Identification An Schorfheide mcmc method}
\label{table:tbl:WeakAnScho_mcmc}
\end{longtable}
\end{document}
