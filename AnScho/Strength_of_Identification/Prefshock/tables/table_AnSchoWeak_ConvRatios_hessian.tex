\documentclass[a4paper,10pt]{article}
\usepackage{amsmath,amsfonts}
\usepackage[left=1cm,right=1cm,top=1cm,bottom=1cm]{geometry}
\usepackage{longtable}
\usepackage{booktabs}
\begin{document}
\centering
\begin{longtable}{cccccccccccccccc}
\toprule
 & RA & PA & GAMQ & TAU & NU & PSIP & PSIY & RHOR & RHOG & RHOZ & SIGR & SIGG & SIGZ & RHOZETA & SIGZETA \\
\midrule
300/100 & 4.909 & 7.308 & 6.130 & 4.096 & 2.700 & 2.938 & 3.372 & 2.999 & 4.148 & 3.004 & 3.513 & 3.318 & 2.894 & 8.087 & 2.021 \\
900/300 & 4.172 & 2.441 & 2.691 & 9.882 & 5.931 & 2.235 & 2.067 & 3.627 & 4.687 & 3.760 & 3.187 & 3.554 & 5.503 & 1.535 & 8.810 \\
2700/900 & 0.538 & 0.452 & 0.484 & 0.162 & 0.504 & 0.383 & 0.161 & 0.630 & 2.279 & 2.094 & 1.756 & 1.477 & 1.098 & 0.163 & 0.110 \\
8100/2700 & 80.444 & 139.899 & 54.830 & 81.476 & 18.654 & 13.625 & 10.613 & 5.386 & 4.304 & 3.469 & 4.066 & 5.891 & 6.685 & 27.091 & 36.396 \\
\bottomrule
\caption{Bayesian Weak Identification An Schorfheide Convergence Ratioshessian method}
\label{table:tbl:WeakAnSchoConvergenceRatios_hessian}
\end{longtable}
\end{document}
