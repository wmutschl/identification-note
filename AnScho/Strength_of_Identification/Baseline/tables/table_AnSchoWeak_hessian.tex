\documentclass[a4paper,10pt]{article}
\usepackage{amsmath,amsfonts}
\usepackage[left=1cm,right=1cm,top=1cm,bottom=1cm]{geometry}
\usepackage{longtable}
\usepackage{booktabs}
\begin{document}
\centering
\begin{longtable}{cccccccccccccc}
\toprule
 & RA & PA & GAMQ & TAU & NU & PSIP & PSIY & RHOR & RHOG & RHOZ & SIGR & SIGG & SIGZ \\
\midrule
100 & 0.075 & 0.046 & 0.509 & 0.076 & 12.169 & 0.523 & 0.383 & 12.342 & 3.124 & 37.208 & 35.168 & 4.421 & 9.189 \\
300 & 0.045 & 0.026 & 0.257 & 0.044 & 11.300 & 0.204 & 0.111 & 9.046 & 9.144 & 25.561 & 31.389 & 5.657 & 8.828 \\
900 & 0.701 & 0.445 & 2.772 & 0.237 & 30.308 & 0.573 & 0.517 & 12.551 & 8.472 & 31.138 & 40.007 & 5.756 & 10.324 \\
2700 & 0.023 & 0.012 & 0.131 & 0.026 & 6.633 & 0.043 & 0.014 & 3.376 & 8.438 & 31.793 & 25.233 & 5.862 & 8.932 \\
8100 & 1.062 & 0.522 & 2.567 & 0.201 & 27.107 & 0.825 & 1.069 & 17.948 & 7.584 & 68.827 & 42.552 & 7.818 & 14.996 \\
\bottomrule
\caption{Bayesian Weak Identification An Schorfheide hessian method}
\label{table:tbl:WeakAnScho_hessian}
\end{longtable}
\end{document}
