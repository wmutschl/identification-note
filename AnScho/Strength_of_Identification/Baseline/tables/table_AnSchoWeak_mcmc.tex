\documentclass[a4paper,10pt]{article}
\usepackage{amsmath,amsfonts}
\usepackage[left=1cm,right=1cm,top=1cm,bottom=1cm]{geometry}
\usepackage{longtable}
\usepackage{booktabs}
\begin{document}
\centering
\begin{longtable}{cccccccccccccc}
\toprule
 & RA & PA & GAMQ & TAU & NU & PSIP & PSIY & RHOR & RHOG & RHOZ & SIGR & SIGG & SIGZ \\
\midrule
100 & 0.073 & 0.045 & 0.495 & 0.081 & 11.648 & 0.503 & 0.396 & 12.218 & 3.386 & 35.239 & 36.472 & 4.059 & 9.896 \\
300 & 0.042 & 0.023 & 0.233 & 0.044 & 10.966 & 0.212 & 0.103 & 8.774 & 8.550 & 25.330 & 31.041 & 5.302 & 8.838 \\
900 & 0.024 & 0.013 & 0.151 & 0.025 & 6.222 & 0.119 & 0.045 & 6.610 & 9.153 & 26.105 & 33.604 & 5.510 & 8.101 \\
2700 & 0.023 & 0.012 & 0.131 & 0.025 & 6.230 & 0.041 & 0.014 & 3.262 & 8.136 & 28.803 & 25.203 & 5.578 & 8.970 \\
8100 & 0.021 & 0.012 & 0.126 & 0.025 & 6.745 & 0.015 & 0.005 & 1.360 & 8.067 & 28.419 & 13.812 & 5.578 & 9.217 \\
\bottomrule
\caption{Bayesian Weak Identification An Schorfheide mcmc method}
\label{table:tbl:WeakAnScho_mcmc}
\end{longtable}
\end{document}
