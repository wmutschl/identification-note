\documentclass[a4paper,10pt]{article}
\usepackage{amsmath,amsfonts}
\usepackage[left=1cm,right=1cm,top=1cm,bottom=1cm]{geometry}
\usepackage{longtable}
\usepackage{booktabs}
\begin{document}
\centering
\begin{longtable}{cccccccccc}
\toprule
 & ALPHA & RA & DELTA & RHOA & SIGA & THETA & KAPPA & RHOUPSILON & SIGUPSILON \\
\midrule
300/100 & 1.910 & 0.988 & 1.193 & 2.096 & 2.186 & 0.950 & 1.719 & 2.324 & 0.693 \\
900/300 & 1.735 & 1.004 & 2.178 & 2.657 & 2.741 & 1.011 & 1.078 & 2.632 & 1.365 \\
2700/900 & 1.476 & 1.004 & 1.364 & 2.950 & 1.936 & 1.008 & 0.855 & 2.830 & 1.284 \\
8100/2700 & 1.210 & 1.020 & 1.781 & 2.997 & 3.024 & 1.042 & 1.027 & 2.980 & 1.220 \\
\bottomrule
\caption{Bayesian Weak Identification An Schorfheide Convergence Ratioshessian method}
\label{table:tbl:WeakAnSchoConvergenceRatios_hessian}
\end{longtable}
\end{document}
